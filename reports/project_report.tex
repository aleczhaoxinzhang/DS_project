\documentclass{article}
\usepackage{booktabs}
\usepackage{amsmath}
\usepackage{float}
\usepackage{caption}

% Language setting
% Replace `english' with e.g. `spanish' to change the document language
\usepackage[english]{babel}

% Set page size and margins
% Replace `letterpaper' with`a4paper' for UK/EU standard size
\usepackage[letterpaper,top=2cm,bottom=2cm,left=3cm,right=3cm,marginparwidth=1.75cm]{geometry}

% Useful packages
\usepackage{amsmath}
\usepackage{graphicx}
\usepackage[colorlinks=true, allcolors=blue]{hyperref}

\title{Rediscover Predictability: Information from the Relative Prices of Long-term and Short-term Dividends}
\author{Alec Zhang, Zihao Liu, Yu Guo, Coco Qu}

\begin{document}
\maketitle

\begin{abstract}\noindent
In this project, we endeavor to replicate and validate the 
findings of Li Wang (2023) in the study ``Rediscover Predictability: 
Information from the Relative Prices of Long-term and Short-term 
Dividends.'' The core objective is to reproduce the results 
presented in Table 1, which provides a statistical summary of the 
price ratio (\( pr_t \)) and the log price-dividend ratio (\( pd_t \))
, and Table 2, which details the regression outcomes predicting the 
12-month returns of the S\&P 500 index using \( pr_t \), \( pd_t \), 
and their residuals.
\newline
\newline
The replication employs data from Bloomberg for S\&P 500 futures and 
utilizes the Fama-Bliss database to closely follow the original study's 
methodology. We reconstruct the statistical relationships and analyze 
the regression coefficients (\( \beta \)) and the adjusted R-squared 
values to assess the predictability of stock returns based on the 
relative pricing of dividends. Our efforts focus on matching the 
precision of the original work, acknowledging the absence of standard
 errors in our regression outputs.
\end{abstract}

\section{Introduction}
This project replicates the pivotal findings in the article. Our 
objective is to replicate Table 1 and Table 2. Utilizing Bloomberg 
data for S\&P 500 futures and zero-coupon yields from the Federal 
Reserve, we not only include analysis from January 1988 to June 
2017 time frame, but also include data up to January 2024. This 
replication effort not only underscores the importance of empirical 
validation in financial research but also confronts the challenges 
of data availability and methodological adaptation inherent in such 
scholarly pursuits.

\section{Replication Tables}
Table 1 is a statistical summary of \( pr_t \), \( pd_t \). \( pr_t \) 
is price ratio of long- to short-term dividend prices, which is calculated by 
\begin{equation}
pr_t = ln \left( \frac{\text{Price of Long-term Dividends}}{\text{Price of Short-term Dividends}} \right) = \ln \left( \frac{P^{T+}}{P^{T-}} \right)
\end{equation}
where, \(P^{T+}\) is the multiplication of price of ZCB (zero coupon bonds) 
that matures in T periods and futures price that is the Q-expectation of 
future stock price. \(P^{T-}\) is the portion of the index value that is not 
accounted for by the futures component, which is S\&P 500 stock prices minus \(P^{T+}\)
\newline
\newline
\( pd_t \) is log price-dividend ratio, which is calculated by 
\begin{equation}
pd_t = \ln \left(\frac{P_t}{D_t} \right)
\end{equation}
Here is the replicated Table1 in the article
\begin{table}[H]
\centering
\input{../output/table_1.tex}
\caption*{Table 1: Summary Statistics}
\label{tab:your_label}
\end{table}

Here is the Table1 with updated numbers, which means using data until 2024-01-31
\begin{table}[H]
\centering
\input{../output/table_1_curr.tex}
\caption*{Table 1: Updated Summary Statistics}
\label{tab:your_label}
\end{table}

Table 2 reports the results of the regression that predicts 
the 12-month S\&P 500 index return via \( pr_t \), \( pd_t \), and their 
residuals after projecting on them. 
\newline
\newline
Here is the replicated Table2 in the article
\begin{table}[H]
\centering
\input{../output/table_2.tex}
\caption*{Table 2: One-year Return Prediction}
\label{tab:your_label}
\end{table}    

Here is the Table2 with updated numbers, which means using data until 2024-01-31
\begin{table}[H]
    \centering
    \input{../output/table_2_curr.tex}
    \caption*{Table 2: Updated One-year Return Prediction}
    \label{tab:your_label}
    \end{table}   


\section{Additional Summary Statistics}
We also created summary statistics similar to Table1 for five source data: 
S\&P500 index, dividend yield, 1-year future price, 1-year 
zero-coupon yields, and the corresponding discounting factor.

\begin{table}[H]
\centering
\input{../output/summary_stats.tex}
\caption*{Table 3: Summary Statistics for Source Data}
\label{tab:your_label}
\end{table}    


\section{Data Sources}
We calculated \(pr_t\) using the S\&P500 indexes and its 1-year expiry 
future prices extracted from Bloomberg, as well as the 1-year zero-coupon 
yields extracted from the Federal Reserve’s website. \(pd_t \)is 
computed via the S\&P500 dividend yield from Bloomberg. These are 
monthly data from Jan 1988 to Jun 2017 for replication and Jan 1988 
to Jan 2024 for creating up-to-date tables.

\section{Successes and Challenges}
In our endeavor to replicate the seminal study, we successfully 
adhered to the paper's logic and replication methodology. Our access 
to and automation of data extraction were particularly noteworthy; 
we were able to seamlessly retrieve data, mirroring the process utilized 
by the original authors. Moreover, our replication of $pd_t$ and $pr_t$ 
revealed only minimal discrepancies in comparison to the published results. 
Significantly, we achieved complete automation of the replication and 
table generation processes, all operationalized through a single 'doit' 
command. The thoroughness of our replication was affirmed by the 
successful passage of all unit tests within reasonable tolerance levels.
\newline
\newline
Despite these successes, we encountered several obstacles. Most notably, 
we were unable to access the Fama-Bliss database, which was the 
source of the 1-year zero-coupon yields in the original research. To 
circumvent this, we sourced the data from the Federal Reserve's website, 
although this introduced some methodological differences. The yields 
from the Federal Reserve are spot rates, which may not align with the 
non-standard rates used in the Fama-Bliss database, potentially leading 
to discrepancies in our calculated $pr_t$ and the ultimate findings. 
Furthermore, discrepancies were noted in the Beta values of the 
replicated table compared to those reported in the original paper.






\end{document}
